\documentclass{article}

% Acronym management: package ``glossaries''
\usepackage[acronym,shortcuts]{glossaries}

% Define your acronyms: ID, short version, long version
\newacronym{WUR}{WUR}{Wageningen University \& Research}
\newacronym{WU}{WU}{Wageningen University}
\newacronym{WR}{WR}{Wageningen Research}
\newacronym{WENR}{WENR}{Wageningen Environmental Research}

% Create an index at the end of the document, simple style
\makenoidxglossaries

\title{Acronyms}
\author{Dainius Masili\=unas}

\begin{document}

\maketitle

\ac{WUR} is an organisation that is comprised of \ac{WU} and \ac{WR}. \ac{WR} includes institutes like \ac{WENR}. While \ac{WENR} is not part of \ac{WU}, they are both part of \ac{WUR} and thus there is a lot of collaboration between them.

% Place the index here
\printnoidxglossary[type=acronym] 

\end{document}
