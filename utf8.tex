\documentclass{article}

% Package for CJK font support in XeLaTeX
\usepackage{xeCJK}

% Package for printing ``XeLaTeX''
\usepackage{hologo}

% Package for stacking of characters for ruby
\usepackage{mathtools}

\title{中文}
\author{Overleaf developers and Dainius Masiliūnas}

\begin{document}

\maketitle

\section{前言}
這是一些文字。
這是繁體中文。

Math works as usual:

$$ e = mc^2 $$

\section{关于数学部分}
数学、中英文皆可以混排。You can intersperse math, Chinese and English (Latin script) without adding extra environments. Note that this requires compiling with \hologo{XeLaTeX} or \hologo{LuaTeX}.

\section{Ruby}

You can also write ruby (also known as $\stackrel{\text{furigana}}{\text{ふりがな}}$), where e.g. $\stackrel{\text{pīnyīn}}{\text{拼音}}$ is written at the top and $\stackrel{\text{hànzì}}{\text{漢字}}$ is at the bottom.

We can make it more concise by defining our own command.
\newcommand{\ruby}[2]{\stackrel{\text{#1}}{\text{#2}}}
That way, we can just use \texttt{\textbackslash{ruby}} like this: $\ruby{Three Kingdoms}{三國}$ is a famous time period, known for its heroes $\ruby{Cao}{曹}$ $\ruby{Cao}{操}$, $\ruby{Liu}{劉}$ $\ruby{Bei}{備}$ and $\ruby{Sun}{孫}$ $\ruby{Quan}{權}$.

\end{document}
